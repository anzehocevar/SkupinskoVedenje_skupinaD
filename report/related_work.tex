\cite{lin2025group}~researched the effect of perturbations on the behavior of schools of fish based on the burst-and-coast model.
The perturbations were modeled as a slight modification of the attraction and alignment strength of a subset of fish in the population.
Their findings showed that larger groups of fish (N = 100) are much more sensitive to perturbations than smaller groups (N = 25 or 50) across most combinations of attraction and alignment strength.

The paper we're expanding upon is based on an earlier research paper~\cite{calovi2018disentangling} that also tackles the burst-and-coast model on the same species of fish.
This paper used raw data, obtained from capturing the movement of fish in enclosed tanks using a digital camera, along with domain knowledge about the specific species of fish, to construct equations that govern the movement of fish during the kick and glide phases.
They also separated the drives of avoiding the wall of the tank and interacting with another (neighboring) fish.