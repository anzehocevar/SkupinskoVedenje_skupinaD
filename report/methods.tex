We re-implemented the model from the original paper~\cite{Wang2025_fish} in Python.
We then extended it with two additional parameters that model a non-zero duty cycle.

\subsection{Base model}

The original burst–coast model consists of a repeated cycle:

\begin{enumerate}
    \item \textbf{Heading selection:}
    Before each burst, the fish computes a new heading based on random noise, attraction and alignment to its $k=1$ or $k=2$ most influential neighbors.
    \item \textbf{Burst phase (kick):} The fish updates its heading and samples a random kick time and length.
    \item \textbf{Coast phase:} The fish now moves along a straight path for the duration and length of its kick, with an exponentially decreasing velocity.
    \item \textbf{Repeat:} Once the fish reaches the end of its coast phase, it immediately selects a new heading and begins a new burst.
\end{enumerate}

\subsection{Extended model}
The extended model introduces two key parameters: the duty cycle $\omega$ and the number of decision steps $n_\omega$. This transforms the motion from discrete jumps to continuous velocity integration, which is a more realistic representation of fish locomotion. 

\subsubsection{Duty cycle}

When a new swimming cycle beings, its total duration $\tau$, is split into burst and a coast phase:

\begin{align*}
    \tau_{\text{burst}} &= \omega \cdot \tau \\
    \tau_{\text{coast}} &= (1 - \omega) \cdot \tau
\end{align*}

At low duty cycles ($\omega\approx0$), we expect the behavior of the extended model should resemble the behavior of the base model. At higher duty cycles, the velocity will not immediately begin to decrease when a new heading is selected.

\subsubsection{Number of decision steps}

The parameters $n_\omega$ controls the granularity of decision making during the burst phase. It dictates how many times a fish re-evaluates social forces and adjust it's direction within a single burst. 

\subsection{Evaluation Metrics}

We use three metrics to evaluate collective behavior.
These will also be used to compare our results with the original model.
The first one is \textit{Group Dispersion}, representing the average square of distance from the barycenter (i.e. how much the fish are spread out in space).
The second one is \textit{Group Polarization}, which is the measure of how varied the headings of different fish are.
Lastly, the \textit{Milling Index} quantifies the degree of how much the fish are swimming around a barycenter in a circular fashion.
We will use the exact same metrics as the original paper with the goal of producing comparable results:

\textbf{Group Dispersion:}
\[
D(t) = \sqrt{\frac{1}{N}\sum_{i=1}^N \|\vec{u}_i(t)-\vec{u}_B(t)\|^2}
\]
where $\vec{u}_i(t)$ refers to position of fish $i$ at time step $t$, $\vec{u}_B(t)$ refers to position of barycentre and $N$ refers to the number of fish.

\textbf{Group Polarization:}
\[
P(t) = \left \| \frac{1}{N} \sum_{i=1}^N 
\frac{v_i(t)}{\|v_i(t)\|} \right\|
\]
where $\vec{v}_i(t)$ refers to the velocity vector of fish $i$ at time step $t$.

\textbf{Milling Index:}
\[
M = \left| \frac{1}{N} \sum_{i=1}^N 
   \sin{(\bar{\theta}_w^i(t))} 
\right|.
\]
where $\bar{\theta}_w^i = \bar{\phi}_i-\bar{\theta}_i$ and $\bar{\phi}_i$ is the angle of the fish's heading and $\bar{\theta}_i$ is the angle of the fish's position, both with respect to the barycenter as the coordinate origin.

These metrics allow for classification of ordered schooling, milling, swarming, and disordered phases.

\subsection{Experiments}

We first verified that our implementation of the base model matches that of the original paper.
We varied the attraction strength ($[0, 0.6]$, discrete step $0.05$) and alignment strength ($[0, 1.2]$, discrete step $0.1$), collecting the averages of all three key metrics over 20 simulations for each set of parameters.
Each simulation had $N=100$ fish and stopped at $200000 = 2000 \times N$ kicks.
We performed this experiment for both $k=1$ and $k=2$.

To study the effects of the new duty cycle ($\omega$) and decision steps ($n_\omega$) parameters,
we collected the three key metrics while varying $\omega$ and $n_\omega$.
Attraction strength was set to $0.22$, alignment strength was set to $0.6$ and $k$ was set to $1$.
We only performed one simulation for each set of parameters with $N=50$, stopping at $50000$ kicks.

We performed one set of experiments with $\omega \in [0.1, 0.9]$ (discrete step $0.1$) and $n_\omega = 5$,
and another with $\omega = 0.5$ and $n_\omega \in [1, 9]$ (discrete step $1$).