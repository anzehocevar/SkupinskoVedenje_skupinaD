The behavior of individual fish in a group is strongly influenced by social interaction,
resulting in collective behaviors such as swarming, schooling and milling.
These behaviors have been extensively studied~\cite{lopez2012behavioural, filella2018model} and reproduced using continuous motion models.

However, continuous motion is not the typical mode of locomotion for many species of fish.
Instead, many species use burst-and-coast swimming~\cite{kramer2001behavioral}
where movement is composed of two cyclical phases:
the burst phase, during which the fish decides on a direction (based on the aforementioned social interactions) and rapidly accelerates towards it;
and the coast phase, during which the fish passively glides and does not actively attempt to change its speed or heading outside of deceleration due to drag.
The ratio between the duration of the burst phase and the total durations of both phases is known as the duty cycle~\cite{chao2024turning}.

We build upon a paper by Wang et al.~\cite{Wang2025_fish} which models the behavior of \textit{Hemigrammus rhodostomus} using agents that implement burst-and-coast swimming and analyzes the long-term collective behavior of large groups of such agents.
One of the simplifications made by the model is to treat the burst phase as an instantaneous event due to its short duration compared to the coast phase~\cite{calovi2018disentangling}, which is equivalent to a duty cycle of 0\%.
We will introduce additional parameters to model a non-zero duty cycle and study its effects on the collective behavior.